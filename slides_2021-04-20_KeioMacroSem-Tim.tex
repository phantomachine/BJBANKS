\documentclass[10pt,english,slidetop,compress,
              blue,mathserif,color=option]{beamer}
\usepackage[T1]{fontenc}
%\usepackage[latin9]{inputenc}
\PassOptionsToPackage{natbib=true}{biblatex}
\usepackage[style=authoryear]{biblatex}
%\usepackage{natbib}
%% Set how deep list environments go
\setcounter{secnumdepth}{3}
\setcounter{tocdepth}{3}

%%==== Table custom
\usepackage{relsize}
\usepackage{rotating}
\usepackage{tablefootnote}
\providecommand{\tabularnewline}{\\}
\usepackage{dcolumn}
\newcolumntype{d}[1]{D..{#1}}
\newcommand{\mc}[1]{\multicolumn{1}{c@{}}{#1}}
%%==== Table custom

\usepackage{babel}
\usepackage{float}
\usepackage{booktabs}
\usepackage{amsmath}
\usepackage{amsthm}
\usepackage{amssymb}
\usepackage{graphicx}
\PassOptionsToPackage{normalem}{ulem}
\usepackage{ulem}

%% Legacy hack from other LyX generated tables
\providecommand{\tabularnewline}{\\}

%% Hyperlink stylin'
\ifx\hypersetup\undefined
  \AtBeginDocument{%
    \hypersetup{unicode=true,pdfusetitle,
 bookmarks=true,bookmarksnumbered=false,bookmarksopen=false,
 breaklinks=false,pdfborder={0 0 0},pdfborderstyle={},backref=slide,colorlinks=true,
 urlcolor=blue,linkcolor=blue,citecolor=red!20!brown!80!black}
  }
\else
  \hypersetup{unicode=true,pdfusetitle,
 bookmarks=true,bookmarksnumbered=false,bookmarksopen=false,
 breaklinks=false,pdfborder={0 0 0},pdfborderstyle={},backref=slide,colorlinks=true,
 urlcolor=blue,linkcolor=blue,citecolor=red!20!brown!80!black}
\fi

\makeatletter

%%%%%%%%%%%%%%%%%%%%%%%%%%%%%% Textclass specific LaTeX commands.
% this default might be overridden by plain title style
\newcommand\makebeamertitle{\frame{\maketitle}}%
% (ERT) argument for the TOC
\AtBeginDocument{%
  \let\origtableofcontents=\tableofcontents
  \def\tableofcontents{\@ifnextchar[{\origtableofcontents}{\gobbletableofcontents}}
  \def\gobbletableofcontents#1{\origtableofcontents}
}
\theoremstyle{plain}
\newtheorem{thm}{\protect\theoremname}
\theoremstyle{definition}
\newtheorem{defn}[thm]{\protect\definitionname}

%%%%%%%%%%%%%%%%%%%%%%%%%%%%%% User specified LaTeX commands.
%\usetheme{Warsaw}
%\usetheme{Boadilla}
\usepackage{threeparttable}
\usepackage{subfigure}
\usepackage{relsize}
\usepackage{url}
% Workaround for Linux: font library problem in Beamer
% See: https://tex.stackexchange.com/questions/383668/mktextfm-ecss0400-fails
\usepackage{lmodern}

\usetheme{default}
\useinnertheme[shadow]{rounded}
\usepackage{beamerinnerthemerounded}
% or ...
\usecolortheme{default}
% color themes: albatross, beaver, beetle, crane, default, dolphin, dov, fly, lily, orchid, rose, seagull, seahorse, sidebartab, structure, whale, wolverine

%\useoutertheme[subsection=false]{smoothbars}
% outer themes include default, infolines, miniframes, shadow, sidebar, smoothbars, smoothtree, split, tree

%\usecolortheme{orchid}
\setbeamertemplate{footline}[text line]{} % makes the footer EMPTY

\setbeamercovered{transparent}
% or whatever (possibly just delete it)

\usepackage{relsize}
\usepackage{tikz, tikzscale}
\usetikzlibrary{shapes,snakes}
\usetikzlibrary{calc}
\usepackage{relsize}
% Adapted from beamerinnerthemedfault.sty
\setbeamertemplate{itemize item}{\scriptsize\raise1.25pt\hbox{\donotcoloroutermaths$\blacktriangleright$}}
\setbeamertemplate{itemize subitem}{\tiny\raise1.5pt\hbox{\donotcoloroutermaths$\blacktriangleright$}}
\setbeamertemplate{itemize subsubitem}{\tiny\raise1.5pt\hbox{\donotcoloroutermaths$\blacktriangleright$}}
\setbeamertemplate{enumerate item}{\insertenumlabel.}
\setbeamertemplate{enumerate subitem}{\insertenumlabel.\insertsubenumlabel}
\setbeamertemplate{enumerate subsubitem}{\insertenumlabel.\insertsubenumlabel.\insertsubsubenumlabel}
\setbeamertemplate{enumerate mini template}{\insertenumlabel}
\setbeamersize{text margin left=20pt,text margin right=20pt}

\usefonttheme[onlymath]{structurebold}

%Color scheme - TK modified
%\setbeamercolor{normal text}{fg=brown!20!black,bg=brown!50!white}
%\setbeamercolor{title}{fg=red!50!black}xxxx
%\setbeamercolor{section in head/foot}{fg=red!40!black,bg=white!80!brown}
%\setbeamercolor{frametitle}{fg=brown!60!orange!50!black}
\setbeamercolor{item}{fg=orange!50!yellow!80!red!20!blue!20!black}  %{fg=orange!50!red!50!brown}
\setbeamercolor{math text}{fg=blue!90!green!60!black} %{fg=brown!50!red!60!black}
\setbeamercolor{alerted text}{fg=red!30!black!60!brown}
\setbeamercolor{quote}{fg=blue!90!green!60!black}
\setbeamercolor{block title}{fg=red!40!black!60!brown}
\setbeamercovered{transparent=25}
\setbeamercolor{button}{fg=orange, bg=gray!30!white}

\makeatletter
\define@key{beamerframe}{wide}[30pt]{%
  \def\beamer@cramped{\itemsep #1\topsep0.5pt\relax}}
\makeatother

%gets rid of bottom navigation bars
\setbeamertemplate{footline}[21]{}

%gets rid of bottom navigation symbols
\setbeamertemplate{navigation symbols}{}

%gets rid of footer
%will override 'frame number' instruction above
%comment out to revert to previous/default definitions
\setbeamertemplate{footline}{}

\makeatother

% Color settings - custom defs
\usepackage{color}
\definecolor{sangre}{rgb}{0.6,0.18,0.19}
\definecolor{dullmagenta}{rgb}{0.4,0,0.4}
\definecolor{darkblueroyale}{rgb}{0,0,0.6}
\definecolor{verdeprofundo}{rgb}{0.02,0.376,0.031}
\definecolor{aqua}{rgb}{0.0, 1.0, 1.0}


\providecommand{\definitionname}{Definition}
\providecommand{\theoremname}{Theorem}

%% ============================================================================
%%    CUSTOMIZABLE: Title Page Information
%% ----------------------------------------------------------------------------
\title[BJBanks]{Money, Credit and Equilibrium Imperfectly Competitive Banking
                % \\ 
                % {\small\emph{Interest-rate Pass Through and Optimal Stabilization Policy}}
                }

\author[Head, Kam, Ng]{
                       \color{yellow!60!red!50!brown!50!black}
                       Allen Head{\color{gray}\inst{1}},
                       Timothy Kam{\color{gray}\inst{2}$^{,}$\inst{3}},
                       Ieng-Man Ng{\color{gray}\inst{2}},
                       Isaac Pan{\color{gray}\inst{4}}
                      }

\institute[~QED-ANU/RSE]{
                    \color{gray!50!black}
                    \inst{1}Queen's University
                    \and
                    \inst{2}Australian National University
                    \and
                    \inst{3}Sungkyunkwan University
                    \and
                    \inst{4}University of Sydney
                    }

\date{ \color{gray!80!blue}

      {\smaller Keio University, Macroeconomics Workshop, April 20, 2021}
      \\
      \bigskip
      \bigskip

      {\smaller\texttt{
      \color{orange!20!brown!50!black}
      \href{https://github.com/phantomachine/BJBANKS}{https://github.com/phantomachine/BJBANKS}
      }}
}

\addbibresource{bjbanks.bib}


\begin{document}
%% ============================================================================
\begin{frame}
  \titlepage
\end{frame}


\section{Introduction}
%% ============================================================================
% \begin{frame}{Background}

%   %Abstract from a multiple-assets real world\footnote{Generalizable: endo-/exo-genous multi-asset portfolio-adjustment cost} ...

%   \begin{block}{}
%     \begin{center}
%       \includegraphics[scale=0.3]{figures/bank-money-krw}
%     \end{center}
%   \end{block}

%   \bigskip

%   Consider an environment with a realistic information friction:

%     \begin{itemize}
%       \item In some markets, agents are ``\alert{anonymous}''

%       \item private securities not sustainable in these markets: private histories, imperfect record keeping, contract unenforceability

%       \item Money is Memory --- equilibrium medium of exchange \citep[][JET]{Kocherlakota1998}

%       \item But self-insurance by carrying money is costly ...
%     \end{itemize}
% \end{frame}

%%===========================================================================
\begin{frame}{Facts: Imperfect Competition in Banking}

  % Susbtantial heterogeneity in banking industry
  %  \bigskip

    

    \begin{itemize}
      
      \item For example, U.S.:

        \begin{itemize}
          \item High profit margins: Markups (90\%)
          \item Imperfect interest-rate pass through: Rosse-Panzar $H$-statistic\footnote{Sum of the elasticities of a bank's total revenue with respect to that bank's input prices} (50\%)
        \end{itemize}

      \medskip

      \item Market share of top-3 banks: Portugal (89\%), Germany (78\%), the United Kingdom (58\%), Korea (1998-2006: 80\%-100\%, post 2007: 50\%), Japan (44\%), United States (35\%)

      \medskip


      \item Data source:
        \begin{itemize}
          \item FDIC, Call Reports, U.S. Commercial Banks, 1984-2010
          \item Bankscope
          \item \citep[][]{CorbaeDerasmo2015, CorbaeDerasmo2018}
        \end{itemize}
    \end{itemize}
\end{frame}

%===========================================================================
\begin{frame}{Monetary Policy and Regulator Concerns}

  % Susbtantial heterogeneity in banking industry
  %  \bigskip

    %\begin{itemize}
      %\item 
      %Monetary-policy transmission effects
      
      %\item 
      
      %Regulatory concerns:

      \begin{itemize}
        \item Rod Sims (Chair of ACCC), 
        {\color{gray}
         \emph{Committee Hansard}, 2016-10-14: 
        }
        \medskip

        \begin{quote}
          There seems a lack of very robust competition in banking \dots 
          %We are not seeing as much robust competition as we would like.
        \end{quote}
      
        \item Australian Productivity Commission Inquiry 
        {\color{gray}
          (No.89, 2018-06-29):
        }
        \medskip

          \begin{quote}
            %High market concentration does not necessarily indicate that competition is weak, that community outcomes will be poor or that structural change is required. Rather, 
            [I]t is the way market participants gain, maintain and use their market power that may lead to poor consumer outcomes.
            \dots
            Reforms that alter \emph{incentives of} ... \emph{banks}, ... aimed directly at bolstering consumer power in markets, and reforms to the governance of the financial system, should be the prime focus of policy action.
          \end{quote}
    \end{itemize}

    \begin{itemize}
      \item Carolyn Wilkins (Asst. Gov. BoC), 
      {\color{gray}
        \emph{Why Do Central Banks Care About Market Power?}, G7 Conference at Banque du France (2019-04-08):
      }
      \medskip

          \begin{quote}
            Still-unanswered questions for central banks about
            implications for people, inflation dynamics, and
            monetary policy transmission.
          \end{quote}
    \end{itemize}
\end{frame}

%%===========================================================================
% \begin{frame}{In Theory ...}

%     \begin{block}{\citet[][JET]{Berentsen2007} ...}

%       \medskip

%       Financial Intermediation (Banks) improve welfare by supporting ``anonymous'' exchange  ...

%       \medskip

%       \hfill so long as holding money is somewhat costly (not at Friedman Rule)

%       \bigskip

%       %\bigskip

%       {%\color{gray}
%       ... or borrowers have incentive constraints (even at Friedman Rule): 

%       \hfill \citet{BoelWaller2019}
%       }

%       % \hfill \citet[][JET]{Berentsen2007}

%     \end{block}


%     \vspace{2cm}

%     \hfill ... Or is it? What if banking industry is \textbf{not perfectly competitive}?

% \end{frame}



%% ============================================================================
\begin{frame}{What We Do}

  

  %\break

  % \begin{block}{Empirical implication}

  %   \bigskip

  %   \begin{itemize}
  %     \item Data: Consumer-loan markups by banks \emph{positively correlated} with the dispersion in loan rate distribution.
      
  %     \bigskip

  %     \hfill We rationalize this.
  %   \end{itemize}

  %     % \begin{center}
  %     %   \includegraphics[scale=0.45]{figures/markup-dispersion-spain}
  %     %   \medskip
  %     % \end{center}
  % \end{block}

  % \break

  % \begin{block}{Empirical, Theoretical/Normative questions}
  %   \bigskip
  
  %   \begin{enumerate}

  %     \item Data: (Consumer) Loan-rate \alert{markups} negatively (\emph{positively correlated}) with \alert{markup dispersion (CV)} (markup s.d.).
  %       \bigskip
    
  %     \item How does \alert{equilibrium market power} of banks distort \emph{basic} intermediation, welfare-enhancing (essential) role of banks?
  %       \bigskip
  
  %     When is financial intermediation (``banking'') \emph{essential}?
  
  %       \bigskip
  
  %     %% SUPPRESS FOR NOW ... no concrete results yet!
  
  %     \item What is an \alert{optimal redistributive liquidity policy}?
  %     %
  %     \bigskip
      
  %     {\color{gray}
  %       \small
  %       \hfill
  %       Irrelevant when no banking market power
  %     }
  %   \end{enumerate}
  % \end{block}

  An equilibrium model of \alert{money} and \alert{credit} with \alert{endogenous market power in lending}.

  \bigskip

  In this environment we ask:

  \bigskip

  How does imperfect competition among lenders affect
  \begin{itemize}
    \item the level and distribution of loan rates,
    \item the pass-through of bank costs to loan rates,
    \item the welfare effects of banking, and
    \item the design of cyclical policy in response to aggregate demand fluctuations?
  \end{itemize}
\end{frame}

%% ============================================================================
\begin{frame}[allowframebreaks]{What We Find}
  \begin{enumerate}
    \item \alert{U.S. Empirical Evidence / Model Validation} ...
    
      \bigskip
    
      Model fitted to macro-data on money demand and average markups
      \begin{itemize}
        %\item 
        \item Predicts positive (negative) correlation between average markups and s.d. (c.v.) of loan rate markups. 
        \medskip
        
        % Micro and macro evidence:
        % \begin{itemize}
        %   \item state level data, controlling for fixed effects$^{**}$
        %   \item national level data$^{**}$
        % \end{itemize}

        % \item Negative correlation between average markups and c.v. of loan rate markups
        
        % \medskip
        
        % \item 
      \end{itemize}

      \bigskip

      Micro and macro evidence:
      \begin{itemize}
        \item state level data, controlling for fixed effects
        \item national level data
      \end{itemize}

      % --------------------------------------------------
      \break
      \item \alert{Pass through} ...
        \medskip

        For low inflation range, cutting inflation target leads to:
        \begin{itemize}
          \item decreased dispersion of lending rates
          \item increased average mark-up
          \item less lending
          \item Higher-then-lower bank expected profits
        \end{itemize}

        \bigskip

       For high inflation range, increasing inflation reduces the real values of loans and profits diminish along with welfare.

      % --------------------------------------------------
      \break

    \item \alert{Essentiality (welfare-enhancing role) of banks} ...
       \bigskip

       The presence of banking may lower welfare:
        \begin{itemize}
          \item Effect is worst at low inflation (away from the Friedman rule)
          \item Also worse when aggregate demand is high
        \end{itemize}

        \bigskip

        \begin{center}
                              \alert{Benefit of banking}: 
                              \\
            Deposits interest on ex-post idle cash (insurance role) 
            \medskip
          
                        $\updownarrow$
            \medskip

                              \alert{Costs}:
                              \\
            Extraction of consumer surplus (lender market power)
        \end{center}


        \break

       Low inflation ...
        \begin{itemize}
          \item Banks exploit more \emph{intensive margin} markup channel
          \item This destroys welfare-gain from intermediation role of banks.
        \end{itemize}
        \medskip

        Higher inflation ...
        \begin{itemize}
          \item Equilibrium loan price dispersion rises
          \item Banks trade-off markup incentives (Intensive Margin) for making more loans (Extensive Margin) $\Rightarrow$ loan market more competitive
        \end{itemize}

        

      % ------------------------------------------------------------
      \break

    \item \alert{Optimal elastic-currency, demand stabilization policies}:
    
      Given a long-run inflation target
      \begin{itemize}
        \item When aggregate demand ``heats up'' \dots
        \item inject state-dependent liquidity (repo agreement)
        \item Policy is beneficial not because it counters inefficient deposit rate movements as in \citet{Berentsen-Waller2011}, but because of its effects on the distribution of markups
        \item Policy (needs to be) non-monotone in demand states --- internalize IM-vs-EM trade-off of endogenously incompetitive banks!
      \end{itemize}
  \end{enumerate}
\end{frame}

%% ==========================================================================
%%      MILESTONE (SECTION TITLE) SLIDE - EMPIRICAL EVIDENCE
%% ==========================================================================
\input{empirics.tex}

%% ==========================================================================
%%      MILESTONE (SECTION TITLE) SLIDE - MODEL
%% ==========================================================================
{
\setbeamercolor{background canvas}{bg=gray!40!black}
  \begin{frame}
    \begin{center}
      \bigskip
      \bigskip

      {\Huge\bfseries{\color{orange}Model}}
      \bigskip

    \end{center}
  \end{frame}
}



%% ==========================================================================
% \begin{frame}[allowframebreaks]{Overview}
  \begin{frame}{Overview}
  % \begin{itemize}
  %   \item Time is discrete and infinite

  %   \medskip

  %   \item Per-period sequential markets \citep[\`{a} la][]{Lagos2005}:
  %     \begin{itemize}
  %       \item Decentralized Market (DM) --- ``special'' good $q$, anonymous goods trades
  %       \item Centralized Market (CM) --- ``general good'' $x$, Walrasian trades
  %     \end{itemize}
  %     Non-storable goods

  %   \medskip

  %   \item Actors:
  %     \begin{itemize}
  %       \item Households, measure 1 (Eat, work, save money, borrow/deposit in DM, repay in CM)
  %       \item Firms of measure 2 (DM and CM, spot production, labor input)
  %       \item Depository Institutions and Lending Banks (take deposits in, credit line to DM consumers)
  %       \item Goverment: monetary and redistributive/liquidity policy
  %     \end{itemize}
  % \end{itemize}

  % \break

  %\begin{block}{}
    \begin{center}
      \includegraphics[scale=0.4]{figures/HKN_timing_graph}
    \end{center}
  %\end{block}
  \hfill {\smaller\color{gray} \citet{Lagos2005}, \citet{Berentsen2007}, \citet{BurdettJudd1983}}


\end{frame}

%% ==========================================================================
\begin{frame}[allowframebreaks]{Model: Households}
  %\framesubtitle{Households}
  % Preference functions
  %   \[
  %     \begin{cases}
  %       u\left(q\right)    & \text{if DM - active shopper}
  %       \\
  %       0                  & \text{if DM - inactive shopper: saver}
  %       \\
  %       U\left(x\right)-h  & \text{if CM}
  %     \end{cases}
  %   \]
  % \begin{itemize}
  %   \item $u\left(q\right)$ utility of special DM good $q$
  %   \item $U\left(x\right)$ utility of general CM good $x$
  %   \item $h$ linear disutility of labor supply in CM
  % \end{itemize}

  % \bigskip

  % \hfill \`{a} la \citet{Berentsen2007}

  % %% ~~~~~~~~~~~~~~~~~~~~~~~~~~~~~~~
  % \break

  \textbf{Second Market (CM)}

  \bigskip

  Household's valuation of initial money balance (plus transfers) $m+\tau_{2}M$,
  credit debt $l$, and deposit holding $d$, is
  \begin{equation*}
    W\left(m+\tau_{2}M,l,d\right)
    =
    \max_{x,h,m_{+1}}
    \left[
      U\left(x\right)-h+\beta V\left(m_{+1}\right)
    \right]
    \label{eq:CM value}
  \end{equation*}
  subject to
  \begin{equation*}
    x+\phi m_{+1}
    =
    wh+\phi\left(m+\tau_{2}M\right)
    +\phi\left(1+i_{d}\right)d-\phi\left(1+i\right)l + \Pi
    \label{eq:CM budget constraint}
  \end{equation*}
  %where $m \gets m+\tau_{2}M$

  % %% ~~~~~~~~~~~~~~~~~~~~~~~~~~~~~~~
  % \break
  % The first-order conditions w.r.t. $x$ and $m_{+1}$, respectively,
  % are
  % \begin{equation}
  % U_{x}\left(x\right)=1\label{eq:CM foc x}
  % \end{equation}
  % and
  % \begin{equation}
  % \phi=\beta V_{m}\left(m_{+1}\right)\label{eq:CM foc m+}
  % \end{equation}
  %
  % \medskip
  %
  % The envelope conditions are
  % \begin{align}
  % W_{m} & =\phi\nonumber \\
  % W_{l} & =-\phi\left(1+i\right)\label{eq:DM marginal value:m,l,d}\\
  % W_{d} & =\phi\left(1+i_{d}\right)\nonumber
  % \end{align}

  %% ~~~~~~~~~~~~~~~~~~~~~~~~~~~~~~~~~~~~~~~
  \break
  \textbf{First Market (DM)}

  % \bigskip
  % \relsize{-1}

  % An ex-ante agent $m$ at the opening of the first market has expected lifetime utility
  \begin{multline*}
  V\left(m\right)
  =
  %\left(1-n\right)
  n
  \left\{
     \underbrace{
       \alpha_{0}B^{0}\left(m\right)
     }_{\text{No bank}}
  \right.
  \\
  \left.
     %
     \qquad\qquad
     +
     \underbrace{
        \alpha_{1}
        \int
        %\int_{\left[\underline{i},\overline{i}\right]}
        B\left(m;i\right)\text{d}F\left(i\right)
      % \right.
      % %\\
      % \left.
        +
      \alpha_{2}
      \int
      %\int_{\left[\underline{i},\overline{i}\right]}
      B\left(m;i\right)
      \text{d}\left[1-\left(1-F\left(i\right)\right)^{2}\right]
    }_{\text{At least 1 bank}}
  \right\}
  \\
  %+n 
  +(1-n)
  W\left(m + \tau_{s}M - d, 0, d\right)
  \label{eq:DM value}
  \end{multline*}
where
  \[
    B\left(m; i\right)
    =
    \max_{q_{b}, l \in [0,\infty]}
    \left\{
      u\left(q_{b}\right)+W\left(m+\tau_{b}M+l-pq_{b},l,0\right)
      :
      pq_{b}\leq m+l+\tau_{b}M
    \right\},
  \]
and,
  \[
    B^{0}\left(m\right)
    =
    \max_{q_{b}}\left[u\left(q_{b}\right)
    +W\left(m+\tau_{b}M-pq_{b},0,0\right): pq_{b}\leq m+\tau_{b}M\right]
  \]

\end{frame}

%% ===========================================================================
\begin{frame}{Firms}

  \begin{itemize}
    \item Second market (CM):
      \begin{itemize}
        \item Firms are perfectly competitive
        \item linear production with labor
        \item Profit-max strategy: $w = 1$
      \end{itemize}

      \bigskip

    \item First market (DM):
        \begin{equation*}
          \label{firmvalDM}
          S(m)
            =
            \max_{q_{s}}
            \left\{
              -c(q_{s})
              + W(m+\tau_{s}M + pq_{s},0,0)
            \right\}.
        \end{equation*}
    % Here $c(q)$ represents the cost of producing quantity $q$ of special goods, where $c(0)=0$, $c_{q}(q) > 0$ and $c_{qq}(q) \geq 0$. The firms' optimal production plan satisfies
    %   \begin{equation}
    %     c_{q}\left(q_{s}\right) = p\phi.
    %     \label{eq:DM optimal production}
    %   \end{equation}
      \begin{itemize}
        \item Walrasian price taking (variation: bargaining)
        \item Cost of producing $q_{s} \mapsto c(q_{s})$
        \item Cost-min strategy: $c'(q_{s}) = \phi p (\equiv 1)$
      \end{itemize}
  \end{itemize}

\end{frame}

%% ===========================================================================
\begin{frame}[allowframebreaks]{Banks}

  Split the banking system into two parts ...

  \bigskip

  Depository Institutions:
  \begin{itemize}
    \item Take deposits from households and commit to paying interest at rate $i_{d}$ in the CM
    \item Lend to both loan agents and (possibly) the foreign capital market
    \item Perfectly competitive in all aspects
  \end{itemize}

  Loan agents:
  \begin{itemize}
    \item Acquire funds from depository institutions and/or the foreign capital market
    \item Post within-period “consumer loan” rates and match randomly with households
    \item Supply loans to meet demand 
    \item Can enforce repayment of loans in the CM
  \end{itemize}

  \break

  \textbf{Loan agents (``banks'')}

  \begin{itemize}
    % \item $i^{d} \equiv \gamma/\beta -1$ be the marginal cost of the Bank (competitive depository insitutions, perfect enforcement assumption)
    \item Ex-ante profit from posting loan price $i$:
      \begin{equation}
      \Pi\left(i\right) %\equiv\Pi\left(i;m,M,\gamma\right)
      =
      %\left(1-n\right)
      n
      \left[
          \alpha_{1}+2\alpha_{2}
          \left(1-F\left(i\right)\right)+\alpha_{2}\zeta\left(i\right)
        \right]
        R\left(i\right)
        \label{eq:BJ Bank profit function}
      \end{equation}
      where
      \begin{align}
        \zeta\left(i\right)
        & =
        \lim_{\varepsilon\searrow0}
        \left\{ F\left(p\right)-F\left(p-\varepsilon\right)\right\} \label{eq:BJ bank measure of visits under duplicate price}
        \\
        R\left(i\right)
        & =
        l^{\star}\left(m;i,p,M,\gamma\right)
        \left[\left(1+i\right)-\left(1+i^{d}\right)\right]
        \label{eq:BJ Bank profit per customer}
      \end{align}

      \begin{itemize}
        % \item $R\left(i\right)$ is profit per loan customer served
        % \item $l^{\star}\left(m;p,M,\gamma\right)$ is the demand for loans from
        % either Case A \eqref{eq:Demand loan, sigma < 1 (when buyer contacts bank)}
        % or Case B \eqref{eq:Demand loan, sigma > 1 (when buyer contacts bank)}
        \item $n\alpha_{2}\zeta\left(i\right)$ is the measure of consumers contacting bank, when consumers also face the same price $i$ from another bank
        %---i.e., there will be a total measure of $2\left(1-n\right)\alpha_{2}\zeta\left(i\right)$ contacting this and the other Bank with identical price $i$
        \begin{itemize}
        \item Customers randomize between them
        \item In equilibrium probability two banks set same price is zero
        \end{itemize}
      \end{itemize}
  \end{itemize}

  % %% ~~~~~~~~~~~~~~~~~~~~~~~~~~~~~~~~~~~
  % \break

  % Keep markup problem manageable (focus on loan side) ...
  % \bigskip

  % Assume cost of funds:
  % \begin{itemize}
  %   \item \alert{Depository institutions}
  %     \begin{itemize}
  %       \item take deposits, lend them to lending agents (the banks)
  %       \item CM: enforce repayment of these loans, commit to return both deposits plus interest to depositor-households
  %       \item these institutions can invest idle funds with ``foreign'' investors, get exogenous return $i_d := \frac{\gamma}{\beta}-1$
  %       \item lending banks will earn positive expected profit as posted loan rates will be bounded below by $i_d$
  %       \item so depositors always guaranteed rate of return $i_d$
  %       \item Note:
  %         \begin{itemize}
  %           \item choice of ``parametrizing'' $i_d$ corresponds to the outcome of a perfectly-competitive banking sector in \citet{Berentsen2007}
  %           \item So our households here are insured against holding idle balances to the same extent as those in \citet{Berentsen2007}
  %         \end{itemize}
  %     \end{itemize}
  % \end{itemize}

  % %% ~~~~~~~~~~~~~~~~~~~~~~~~~~~~~~~~~~~
  % \break

  % Consider the hypothetical bank serving loan customers who only draw
  % one contact with this bank.

  % \bigskip

  % This bank's ``monopoly'' profit function
  % is
  % \[
  % \Pi^{m}\left(i\right)=\left(1-n\right)\alpha_{1}R\left(i\right)
  % \]

  % \begin{lemma}
  % \label{lem:Monopoly profit must be positive}$\Pi^{m}\left(i\right)>0$
  % for $i>i^{d}$.
  % \end{lemma}

  % %\bigksip


  % %% ~~~~~~~~~~~~~~~~~~~~~~~~~~~~~~~~~~~
  % \break

  % Consider a BJ Bank faced with noisy search. The bank's maximal profit
  % is defined as
  % \begin{equation}
  % \Pi^{\star}\equiv\Pi^{\star}\left(m,M,\gamma\right)=\max_{i\in\text{supp}\left(F\right)}\Pi\left(i;m,M,\gamma\right)\label{eq:BJ Bank - maximal profit}
  % \end{equation}

  % \begin{lemma}
  % \label{lem:BJ bank w noisy search profit must be positive}$\Pi^{\star}>0$.
  % \end{lemma}

  % \begin{proof}
  % Since we are restricting to a class of linear pricing rules, then,
  % for any markup over marginal cost $\mu>1$, the profit from positing
  % $i=\mu i^{d}$ is
  % \begin{align*}
  % \Pi\left(\mu i^{d}\right) & =\left(1-n\right)\left[\alpha_{1}+2\alpha_{2}\left(1-F\left(\mu i^{d}\right)\right)+\alpha_{2}\xi\left(\mu i^{d}\right)\right]R\left(\mu i^{d}\right)\\
  %  & >\left(1-n\right)\alpha_{1}R\left(\mu i^{d}\right)=\Pi^{m}\left(\mu i^{d}\right)>0,
  % \end{align*}
  % where $R\left(i\right)=l^{\star}\left(m;i,p,M,\gamma\right)\left[\left(1+i\right)-\left(1+i^{d}\right)\right]$.
  % The last inequality is from Lemma \ref{lem:Monopoly profit must be positive}.
  % From the definition of the max operator in \eqref{eq:BJ Bank - maximal profit},
  % \begin{align*}
  % \Pi^{\star} & =\max_{i\in\text{supp}\left(F\right)}\Pi\left(i\right)\\
  %  & \geq\Pi\left(\mu i^{d}\right)>\Pi^{m}\left(\mu i^{d}\right)>0.
  % \end{align*}
  % \end{proof}


  % % %% ~~~~~~~~~~~~~~~~~~~~~~~~~~~~~~~~~~~
  % % \break

  % \begin{lemma}
  % \label{lem:The-monopoly-price is maximal}The ``monopoly'' price
  % $i^{m}$ is the largest price in the support of $F_{t}$, $\textup{supp}\left(F_{t}\right)$.

  % \bigskip

  % It is unique.
  % \end{lemma}

  % \begin{proof}
  % Suppose there is a $\bar{i}\neq i^{m}$ which is the largest element
  % in $\text{supp}\left(F_{t}\right)$. Then $\Pi^{m}\left(\bar{i}\right)=\left(1-n\right)\alpha_{1}R\left(\bar{i}\right)$.
  % Since $F\left(i^{m}\right)\geq0$ amd $\xi\left(i^{m}\right)\geq0$,
  % then
  % \begin{align*}
  % \Pi\left(i^{m}\right) & =\left(1-n\right)\left[\alpha_{1}+2\alpha_{2}\left(1-F\left(i^{m}\right)\right)+\alpha_{2}\xi\left(i^{m}\right)\right]R\left(i^{m}\right)\\
  %  & \geq\left(1-n\right)\alpha_{1}R\left(i^{m}\right)=\Pi^{m}\left(i^{m}\right)\\
  %  & >\Pi^{m}\left(\bar{i}\right)
  % \end{align*}
  % The last inequality is true by the definition of a monopoly price
  % $i^{m}$. Therefore $\Pi\left(i^{m}\right)>\Pi^{m}\left(\bar{i}\right)$.
  % The equal profit condition would require that, $\Pi^{m}\left(\overline{i}\right)=\Pi^{\star}\geq\Pi^{m}\left(i^{m}\right)$.
  % Therefore $\overline{i}=i^{m}$.
  % \end{proof}

  % %% ~~~~~~~~~~~~~~~~~~~~~~~~~~~~~~~~~~~
  % \break

  % \begin{lemma}
  % \label{lem:BJ Bank Monopoly Profit Function - Concavity Suff Condition}
  % There is a unique monopoly-profit-maximizing price
  % $i^{m}$ that satisfies the first-order condition
  %   \begin{align*}
  %     \frac{\partial\Pi^{m}\left(i\right)}{\partial i}
  %      =
  %     \left(1-n\right)\alpha_{1}
  %     & \left[
  %       \frac{\partial l^{\star}\left(m;i,p,M,\gamma\right)}{\partial i}
  %       \left(1+i\right)
  %       +l^{\star}\left(m;i,p,M,\gamma\right)
  %     \right.
  %     \\
  %     & \left.
  %       \qquad\qquad -\frac{\partial l^{\star}\left(m;i,p,M,\gamma\right)}{\partial i}\left(1+i^{d}\right)
  %     \right]
  %     \\
  %     & =0.
  %   \end{align*}
  % \end{lemma}

  % \break
  %
  % We can prove that:
  % \begin{enumerate}
  %   \item Bank's faced with noisy-search loan customers earn maximal expected profit equal to monopolist's profit
  %
  %   \item Each bank (pricing at each $i \in \text{supp}F(i)$) trades off
  %     \begin{itemize}
  %       \item intensive-margin profit $R(i)$
  %
  %             \medskip
  %
  %             --- against ---
  %
  %             \medskip
  %
  %       \item extensive-margin profit: probability of agents showing up ("queue length") $\alpha_1 + 2\alpha_2 (1-F(i))$
  %     \end{itemize}
  % \end{enumerate}

% \hfill \hyperlink{F-description}{\beamerbutton{Equilibrium $F$ description}}
%
% \hypertarget{BJ Banks tradeoff}{\beamerbutton{Back: BJ Banks tradeoff}}

%   %% ~~~~~~~~~~~~~~~~~~~~~~~~~~~~~~~~~~~
%   \break
%
%   We also know these properties about the distribution of loan rates:
%
%   \begin{lemma}
%   \label{lem:F is continuous}$F_{t}$ is a continuous distribution
%   function.
%   \end{lemma}
%
%   \begin{proof}
%   Suppose there is a $i\in\text{supp}\text{\ensuremath{\left(F_{t}\right)}}$
%   such that $\xi\left(i\right)>0$ and
%   \[
%   \Pi\left(i\right)=\left(1-n\right)\left[\alpha_{1}+2\alpha_{2}\left(1-F\left(i\right)\right)+\alpha_{2}\xi\left(i\right)\right]R\left(i\right).
%   \]
%   $R$ is clearly continuous in $i$. Hence there is a $i'<i$ such
%   that $R\left(i'\right)>0$ and since
%   $\Delta:=R\left(i\right)-R\left(i'\right)<\frac{\alpha_{2}\xi\left(i\right)R\left(i\right)}{\alpha_{1}+2\alpha_{2}}$.
%   Then
%   \begin{align*}
%   \Pi\left(i'\right) & =\left(1-n\right)\left[\alpha_{1}+2\alpha_{2}\left(1-F\left(i'\right)\right)+\alpha_{2}\xi\left(i'\right)\right]R\left(i'\right)\\
%    & \geq\left(1-n\right)\left[\alpha_{1}+2\alpha_{2}\left(1-F\left(i\right)\right)+\alpha_{2}\xi\left(i\right)\right]\left[R\left(i\right)-\Delta\right]\\
%    & \geq\Pi\left(i\right)+\left(1-n\right)\left\{ \alpha_{2}\xi\left(i\right)\left[R\left(i\right)-\Delta\right]-\left(\alpha_{1}+2\alpha_{2}\right)\Delta\right\}
%   \end{align*}
%   The first weak inequality is a consequence of $F\left(i\right)-F\left(i'\right)\geq\xi\left(i\right)$.
%   Since $R\left(i\right)>\Delta$ and $\Delta<\frac{\alpha_{2}\xi\left(i\right)R\left(i\right)}{\alpha_{1}+2\alpha_{2}}$,
%   then the last line implies $\Pi\left(i'\right)>\Pi\left(i\right)$.
%   This contradicts $i\in\text{supp}\text{\ensuremath{\left(F_{t}\right)}}$.
%   \end{proof}
%   %
%   %% ~~~~~~~~~~~~~~~~~~~~~~~~~~~~~~~~~~~
%   \break
%   \begin{lemma}
%   \label{lem: BJ Banks - supp(F) is connected}The support of $F_{t}$,
%   $\textup{supp}\text{\ensuremath{\left(F_{t}\right)}}$, is a connected
%   set.
%   \end{lemma}
%
%   \begin{proof}
%   Pick two prices $i$ and $i'$ belonging to the set $\text{supp}\text{\ensuremath{\left(F_{t}\right)}}$,
%   and suppose that $i<i'$ and $F\left(i\right)=F\left(i'\right)$.
%   The expected profit under these two prices are, respectively,
%   \[
%   \Pi\left(i\right)=\left(1-n\right)\left[\alpha_{1}+2\alpha_{2}\left(1-F\left(i\right)\right)\right]R\left(i\right),
%   \]
%   and,
%   \[
%   \Pi\left(i'\right)=\left(1-n\right)\left[\alpha_{1}+2\alpha_{2}\left(1-F\left(i'\right)\right)\right]R\left(i'\right).
%   \]
%   Since $F\left(i\right)=F\left(i'\right)$, then the first terms in
%   the profit evaluations above are identical:
%   \begin{align*}
%   \left(1-n\right)\left[\alpha_{1}+2\alpha_{2}\left(1-F\left(i\right)\right)\right] & =\left(1-n\right)\left[\alpha_{1}+2\alpha_{2}\left(1-F\left(i'\right)\right)\right].
%   \end{align*}
%   However, since $i$ and $i'$ belonging to the set $\text{supp}\text{\ensuremath{\left(F_{t}\right)}}$,
%   then clearly, $i^{d}<i<i'\leq i^{m}$. For all $i\in\left[i^{d},i^{m}\right]$, $R(i)<R\left(i'\right)$. From these two observations, we
%   have $\Pi\left(i\right)<\Pi\left(i'\right)$. This contradicts the
%   condition that if firms are choosing $i$ and $i'$ from $\text{supp}\text{\ensuremath{\left(F_{t}\right)}}$
%   then $F_{t}$ must be consistent with maximal profit $\Pi\left(i\right)=\Pi\left(i'\right)=\Pi^{\star}$
%   (viz. the equal profit condition must hold).
%   \end{proof}

% \end{frame}
%
% %% ============================================================================
% \begin{frame}{}

  \break

  We can prove that:
  \begin{enumerate}
    \item Bank's faced with noisy-search loan customers earn maximal expected profit equal to monopolist's profit

    \item Each bank (pricing at some $i \sim F$) trades off
      \begin{itemize}
        \item intensive-margin profit $R(i)$

              \medskip

              --- against ---

              \medskip

        \item extensive-margin loss: probability of agents showing up (``queue length'') $\alpha_1 + 2\alpha_2 (1-F(i))$


      \end{itemize}

      \item All earn the same expected profit
      
      \item If $\alpha_{1} \in (0,1)$, there is a unique non-degenerate, posted-loan-rate distribution $F$. This distribution is continuous with connected support:
      \begin{equation*}
        F\text{\ensuremath{\left(i\right)}}
        =
        1-\frac{\alpha_{1}}{2\alpha_{2}}
        \left[
          \frac{R\left(\overline{i}\right)}{R\left(i\right)}-1
        \right],
        \label{eq:BJ Bank - equilibrium F}
      \end{equation*}
      where %
      $
        \textup{supp}\left(F\right)
          =
        \left[\underline{i},\overline{i}\right]
      $

      \item As $\alpha_{1} \rightarrow 0$, lending becomes competitive (Bertrand limiting case): $\max\{i\} \rightarrow i_{d}$ (BCW!)
      
      As $\alpha_{1} \rightarrow 1$, lending becomes monopolistic: $\min\{i\} \rightarrow i^{m}$

  \end{enumerate}


  % \break



  % In the aggregate, banking sector's loan pricing problem summarized by an analytical \emph{distribution of loan rates}:

  % \begin{lemma}
  % \label{prop:F formula}$F_{t}$ is given by the formula:
  %   \begin{equation}
  %   F\left(i\right)
  %   =
  %   1- \frac{\alpha_{1}}{2\alpha_{2}}
  %   \left[
  %     \frac{R\left(i^{m}\right)}{R\left(i\right)}-1
  %   \right],
  %   \label{eq:BJ Bank - equilibrium F}
  %   \end{equation}
  % where $\text{supp}\left(F_{t}\right)=\left[\underline{i}_{t},\overline{i}_{t}\right]$,
  % $R\left(\underline{i}_{t}\right)
  % =
  % \frac{\alpha_{1}}{\alpha_{1}+2\alpha_{2}}R_{t}
  % \left(\overline{i}_{t}\right)$
  % and $\overline{i}_{t}=i_{t}^{m}$.
  % \end{lemma}

% \begin{proof}
% Since $F$ has no mass points by Lemma \ref{lem: BJ Banks - supp(F) is connected},
% and is continuous by Lemma \ref{lem:F is continuous}, then expected
% profit from any $i\in\text{supp}\left(F\right)$ is a continuous function
% over $\text{supp}\left(F\right)$,
% \[
% \Pi\left(i\right)=\left(1-n\right)\left[\alpha_{1}+2\alpha_{2}\left(1-F\left(i\right)\right)\right]R\left(i\right),
% \]
% where the image $\Pi\left[\text{supp}\left(F\right)\right]$ is also
% a connected set. From Lemma \ref{lem:The-monopoly-price is maximal},
% profit is maximized at $\Pi^{m}\left(i^{m}\right)=\left(1-n\right)\alpha_{1}R\left(i^{m}\right)$.
% For any $i\in\text{supp}\left(F\right)$, the induced expected profit
% must also be maximal, i.e.,
% \[
% \Pi\left(i\right)=\left(1-n\right)\left[\alpha_{1}+2\alpha_{2}\left(1-F\left(i\right)\right)\right]R\left(i\right)=\left(1-n\right)\alpha_{1}R\left(i^{m}\right).
% \]
% Solving for $F$ yields the analytical expression \eqref{eq:BJ Bank - equilibrium F}.
% \end{proof}

% \hfill \hyperlink{BJ Banks tradeoff}{\beamerbutton{BJ Banks tradeoff}}
%
% \hypertarget{F-description}{\beamerbutton{$F$ description}}


\end{frame}

%% ==========================================================================
%%      MILESTONE SLIDE - SME
%% ==========================================================================
{
  \setbeamercolor{background canvas}{bg=gray!40!black}
    \begin{frame}
      \begin{center}
        \bigskip
        \bigskip

        {\Huge\bfseries{\color{orange}SME}}
        \bigskip

      \end{center}
    \end{frame}
}

%% ==========================================================================
\begin{frame}[allowframebreaks]{SME: Households}
  \framesubtitle{Household optimizes}

  Assume $\sigma < 1$. Work with stationary variables:
  \begin{itemize}
    \item $\rho := \phi p$
    \item $z := \phi m$
    \item $Z := \phi M$
    \item $\xi := \phi l$
  \end{itemize}

  % %% ~~~~~~~~~~~~~~~~~~~~~~~~~~~~~~~~~~~~~
  % \break

  Then we have the ordering $0  < \tilde{\rho}_i < \hat{\rho}$ and $0 < \hat{i}$:

  %   \begin{equation}
  %     \begin{split}
  %       & 0  < \tilde{\rho}_i < \hat{\rho}
  %       \\
  %       & 0 < \hat{i}
  %     \end{split}
  %     \label{eq:cutoffs - sigma < 1}
  %   \end{equation}

  % where

    \begin{itemize}
      \item Relative price above which DM liquidity not exhausted: $\hat{\rho}
      :=\hat{\rho}(z;Z,\gamma)
      =\left[z+\tau_{b}Z\right]^{\frac{\sigma}{\sigma-1}}$

      \item Relative price below which DM liquidity binds with borrowing top-up: $\tilde{\rho}_i:=\hat{\rho}\left(1+i\right)^{\frac{1}{\sigma-1}}$

      \item Bank-lending rate below which there is borrowing: $\hat{i}=\rho^{\sigma-1}\left[z+\tau_{b}Z\right]^{-\sigma}-1>0$
    \end{itemize}

  % %% ~~~~~~~~~~~~~~~~~~~~~~~~~~~~~~~~~~~~~
  % \break

  % Optimal DM goods demand functions are:

  % \begin{itemize}

  %   \item For events with probability measure $\alpha_0$
  %       \begin{equation}
  %       q_{b}^{0, \star}\left(z;\rho,Z,\gamma\right)=
  %       \begin{cases}
  %       \frac{z+\tau_{b}Z}{\rho} & \text{if } \rho \leq \hat{\rho}
  %       \\
  %       \rho^{-\frac{1}{\sigma}} & \text{if } \rho \geq \hat{\rho}
  %       \end{cases}
  %       \end{equation}

  %   \item For events with probability measure $\alpha_1$ and $\alpha_2$
  %       \begin{equation}
  %       q_{b}^{\star}\left(z;\rho,Z,\gamma\right) =
  %       \begin{cases}
  %       \left[\rho\left(1+i\right)\right]^{-\frac{1}{\sigma}} & \text{if }
  %       0 < \rho \leq \tilde{\rho}_i \text{ and } 0 \leq   i < \hat{i}
  %       \\
  %       \frac{z+\tau_{b}Z}{\rho} & \text{if }\tilde{\rho}_i<\rho<\hat{\rho}\text{ and } i \geq \hat{i}
  %       \\
  %       \rho^{-\frac{1}{\sigma}} & \text{if } \rho\geq\hat{\rho} \text{ and } i \geq \hat{i}
  %       \end{cases}
  %       \end{equation}
  % \end{itemize}

    %% ~~~~~~~~~~~~~~~~~~~~~~~~~~~~~~~~~~~~~
    \break

    ... and optimal DM loan demand is:

      \begin{equation}
        \xi^{\star}\left(z;i,\rho,Z,\gamma\right) =
        \begin{cases}
          \rho^{\frac{\sigma-1}{\sigma}}\left(1+i\right)^{-\frac{1}{\sigma}}-\left(z+\tau_{b}Z\right) &
          0 < \rho \leq \tilde{\rho}_i \text{ and } 0 \leq   i < \hat{i}
          \\
          0 & \tilde{\rho}_i<\rho<\hat{\rho}\text{ and } i \geq \hat{i}
          \\
          0 & \rho\geq\hat{\rho} \text{ and } i \geq \hat{i}
        \end{cases}
      \end{equation}

    %% ~~~~~~~~~~~~~~~~~~~~~~~~~~~~~~~~~~~~~
    \break

    There is an equilibrium upper and lower bound on the support of the equilibrium loan interest-rate distribution $F$:

    \begin{itemize}
      \item $\overline{i} := \min \left\{ i^{m}, \hat{i}\right\}$
      \item $\underline{i} > \gamma/\beta - 1$
    \end{itemize}

    where

    \begin{itemize}
      \item $i^{m}$ is a well-defined monopoly price
      \item $\underline{i} < \overline{i} \leq i^{m}$
    \end{itemize}


    %% ~~~~~~~~~~~~~~~~~~~~~~~~~~~~~~~~~~~
    \break
    %\textbf{Optimal money demand (special cases detour) ...}

    \bigskip

    \alert{Perfect Competition (BCW)}: 
    \begin{equation}
        \relsize{-1}
        \begin{split}
            \underbrace{
            \frac{\gamma-\beta}{\beta}
            }_{\text{MC of extra dollar}}
            &=
            \underbrace{
                (1-n)i_{d}
                }_{\text{\color{magenta}[A]: MB, idle funds}}
            + 
            \underbrace{
                n i
                }_{\text{\color{magenta}[B]: MB, less borrowing, PC } (i \nwarrow i_d) }
            \\
            & \equiv
            i
        \end{split}
    \end{equation}

    %\bigskip

    \alert{No-bank, self-insurance (BCW, us)}: 
    \begin{equation}
        \relsize{-1}
        \begin{split}
            \underbrace{
            \frac{\gamma-\beta}{\beta}
            }_{\text{MC of extra dollar}}
            &=  
            \underbrace{
                n[u'(q_b)-1]
                }_{\text{\color{magenta}[C]: MB, liquidity premium}}
        \end{split}
    \end{equation}

    \begin{block}{Perfect-competition banking}
      If money yields lower return than other risk-free assets (not at Friedman rule) ...
      
      \hfill BCW: Banks always improve on allocations/trade and thus welfare.
    \end{block}

    %% ~~~~~~~~~~~~~~~~~~~~~~~~~~~~~~~~~~~
    \break


    \alert{Our setting, non-degenerate $F$}: Consider equilibria with positive \emph{ex-post} loans demand and \emph{ex-ante} money demand ($\alpha_0 \neq 0$)  ...
    \bigskip
    
    Optimal money demand satisfies Euler functional equation:

    % \begin{equation}
    % \relsize{-1}
    % \begin{split}
    %  \underbrace{
    %     \frac{\gamma-\beta}{\beta}
    %     }_{\text{MC of extra dollar}}
    %   &=
    % %   \underbrace{
    %     \underbrace{(1-n)i_{d}}_{\color{red!50!blue}\text{[A*]: MB of idle funds (BCW, PC)}}
    % \\
    %   &+
    %   \underbrace{
    %     {\color{verdeprofundo}
    %       %(1-n)
    %       n \int_{i_{\min}}^{i_{\max}}
    %       \mathbb{I}_{\{0 \leq \rho < \tilde{\rho}_i\}}
    %       \left[\alpha_{1}+2\alpha_{2}\left(1 - F\left(i\right)\right)\right]
    %       i \text{d}F(i)
    %     } % color
    %   }_{\text{\color{red!50!blue}[B*]: Borrow, MB less borrowing } \leq ni \text{ (BCW, PC)} }
    % \end{split}
    % \label{eq:z Euler}
    % \end{equation}
    \begin{equation}
      \begin{split}
            %\relsize{-1}
            1 
            =  
            &
            \underbrace{\frac{\alpha_{0}\left(u^{'}[q_{b}^{0}(z)]-1\right)}{i_{d}}}_{\text{Ex-ante self-insurance}} 
            \\
            &
            +
            \underbrace{
                %{\color{verdeprofundo}
                  %(1-n)
                  \int_{\underline{i}(z)}^{\overline{i}(z)}
                  \mathbb{I}_{\{0 \leq \rho < \tilde{\rho}_i\}}
                  \underbrace{
                    \left[
                        \alpha_{1}+2\alpha_{2}\left(1 - F\left(i;z,\gamma\right)\right)
                    \right]
                  }_{\text{Extensive margin}}
                  \underbrace{
                      \left(\frac{i}{i_d}\right)
                  }_{\text{Intensive margin}}
                  \text{d}F(i;z,\gamma).
                %} % color
            }_{\text{Ex-ante markup}}
      \end{split}
      \label{eq:z Euler-with-credit: asset pricing}
    \end{equation}

%     %% ~~~~~~~~~~~~~~~~~~~~~~~~~~~~~
%     \break

%     % Simplifies to:
%     % \begin{equation}
%     %     \relsize{-1}
%     %     1 = 
%     %     \underbrace{
%     %         {\color{verdeprofundo}
%     %           %(1-n)
%     %           \int_{i_{\min}}^{i_{\max}}
%     %           \mathbb{I}_{\{0 \leq \rho < \tilde{\rho}_i\}}
%     %           \underbrace{
%     %             \left[
%     %                 \alpha_{1}+2\alpha_{2}\left(1 - F\left(i\right)\right)
%     %             \right]
%     %           }_{\text{Extensive margin}}
%     %           \underbrace{
%     %               \left(\frac{i}{i_d}\right)
%     %           }_{\text{Intensive margin markup}}
%     %           \text{d}F(i)
%     %         } % color
%     %     }_{\text{Ex-ante markup from h/hold perspective}}
%     % \end{equation}

% %\break

% \begin{block}{Proposition (Banks can be inessential)}

%     At certain $\tau_b$ and thus $F(z, \tau_{b})$, nett MB [A*]+[B*] can be less than (equal to) MB of self-insurance world, i.e., $n[u'(q_b)-1]$.

%     \begin{itemize}
%         \item In calculus of intertemporal money demand, household anticipates bank's ex-post markup vs. matching probability trade-off ... 
%         \item (Earlier) banks' ex-ante profit encodes this too ...
%     \end{itemize}
% \end{block}

\end{frame}

%% ==========================================================
\begin{frame}{SME: Banks}
\framesubtitle{Banks optimize}


  Distribution of loan rates $F$:
    \begin{equation}
        F\left(i;z,\rho,Z,\gamma\right)
        = 1 - \frac{\alpha_1}{2\alpha_2}\left[\frac{R\left(i^{m}\right)}{R\left(i\right)}-1\right],
    \end{equation}

  \begin{itemize}
    \item $\text{supp}\left(F\right)=\left[i_{\min},i_{\max}\right]$

    \item given monopoly price $i^{m}$ and max. willing to pay $\hat{i}$, $i_{\min}$ solves:

      \begin{equation}
          R\left(i_{\min}\right)
          =
          \frac{\alpha_{1}}{\alpha_{1}+2\alpha_{2}}R\left(i_{\max}\right)
      \end{equation}

    where

      \begin{equation}
          R\left(i\right)
          \equiv R\left(i;z,\rho,Z,\gamma\right)
          = \left[\rho^{\frac{\sigma-1}{\sigma}}\left(1+i\right)^{-\frac{1}{\sigma}}
          -\left(z+\tau_{b}Z\right)\right]\left( i-i^{d} \right)
      \end{equation}

      is (real) bank profit per customer served
  \end{itemize}
\end{frame}

%% ==========================================================
\begin{frame}{SME: Firms and Markets}
\framesubtitle{Firms optimize, goods markets clear, loans feasible}

  DM sellers optimize and the Walrasian price-taking DM market clears:

    \begin{equation}
      \begin{split}
        q_{s} \left(z,Z,\gamma\right)
        & \equiv c'^{-1}(\rho)
        \\
        &=
        %\left(1-n\right)
        n
          \alpha_{0}q_{b}^{0,\star}\left(z;\rho,Z,\gamma\right)
        \\
        &+n %\left(1-n\right)
          \left[\int_{\underline{i}}^{\overline{i}}
            \left[
              \alpha_{1}+2\alpha_{2}
              -2\alpha_{2}F\left(i\right)
            \right]
            q_{b}^{\star}\left(z;\rho,Z,\gamma\right)\text{d}F\left(i\right)
        \right]
      \end{split}
      \label{eq:q clearing}
    \end{equation}
  (CM also clear ...)

  \medskip

% \end{frame}

% %% ==========================================================
% \begin{frame}{Stationary Monetary Equilibrium}
% \framesubtitle{Feasibility}

  Total deposits weakly exceed total loans:
    \begin{equation}
      \begin{split}
        (1-n)\delta^{\star}\left(z,Z,\gamma\right)
        & \equiv
        (1-n) \left(\frac{z+\tau_{b}Z}{\rho}\right)
        \\
        & \geq
        n %\left(1-n\right)
        \left\{
          \int_{i_{\underline{i}}}^{\overline{i}}
            \left[
              \alpha_{1}+2\alpha_{2}
              - 2\alpha_{2}F\left(i\right)
            \right]
            \xi^{\star}\left(z;i,\rho,Z,\gamma\right)
            \text{d}F\left(i\right)
        \right\}
      \end{split}
      \label{eq:loan feasibility}
    \end{equation}

\end{frame}

%% ==========================================================
\begin{frame}{Unique SME w/ Money and Banking}

  \begin{block}{Proposition}
    Assume loan contracts are perfectly enforceable. If $\gamma>\beta$, $0 < z^{\star} < \left(\frac{1}{1+\overline{i}(z^{\star})}\right)^\frac{1}{\sigma}$, and $n$ satisfies an endogenous lower bound such that $n \geq N(z^{\star}) \in [0,1]$ and $z^{\star}$, then there exists a unique SME with co-existing money and credit. 
  \end{block}

\end{frame}

%% ==========================================================
\begin{frame}{SME: Market Power and Inflation Targeting}

    \begin{block}{Lemma}
        In an SME, $F(z')$ stochastically dominates $F(z)$, for $z' < z$.
    \end{block} 

    \bigskip

    \begin{block}{Proposition}

        For high-enough long-run inflation target $\tau$ (or $\gamma$) ...
        \begin{itemize}
            \item Real money demand $z$ falls.
            \item Stochastic dominance result $\Rightarrow$ agent-$z$ more likely to draw lower ex-post markups.
            \item Banks tend to care more about customers showing up, mark up less (i.e., tends toward Bertrand competition).
        \end{itemize}
    \end{block}

    \bigskip

    \begin{block}{Proposition}
        Under regularity conditions---equilibrium support not too wide and min loan rate not too high above cost of funds---equilibrium \emph{average markup falls with inflation}.
    \end{block}

\end{frame}



% %% ==========================================================================
% \begin{frame}[allowframebreaks]{Unique SME}
%
%   \bigskip
%
%   \begin{theorem}
%     Given monetary policy $\tau$, there is a unique stationary monetary equilibrium.
%   \end{theorem}
%
%   \bigskip
%
%   \begin{center}
%     \includegraphics[scale=0.5]{figures/Unique_SME_sigma_lt_one}
%   \end{center}
%
%   \break
%
%   % ~~~~~~~~~~~~~~~~~~~~~~~~~~~~~~~~~~~~~~~~~~~~~~~~~~~~~~~~~~~~~~~~~~
%   \begin{proof}
%     \begin{itemize}
%       \item Inspection of SME conditions show problem boils downs to money demand characterization (Euler functional equation)
%       \item LHS of Euler is marginal cost of holding an extra real dollar at end of CM, $z$ --- constant for given inflation rate
%       \item RHS is net marginal benefit of carrying that extra dollar into the next DM market (and all probable continuation marginal valuations thereafter)
%       \item Show that RHS is monotone decreasing in $z$:
%         \begin{itemize}
%         \item First-order stochastic dominance $F(\cdot, z) \prec F(\cdot, z')$ for $z > z'$
%         \item Diminishing marginal benefit w.r.t. $z$
%         \item Endogenous integral cut-offs shrink with $z$
%         \end{itemize}
%     \end{itemize}
%   \end{proof}
%
% \end{frame}

%% ==========================================================================
%%      MILESTONE SLIDE - DATA VALIDATION
%% ==========================================================================
{
\setbeamercolor{background canvas}{bg=gray!40!black}
  \begin{frame}
    \begin{center}
      \bigskip
      \bigskip
      {\Huge\bfseries{\color{orange}Empirical Validation}}
      \bigskip

    \end{center}
  \end{frame}
}

% %% ==========================================================================
% \begin{frame}{Benchmark Calibration}
%   \input{calibration.tex}
% \end{frame}

%% ==========================================================================
\begin{frame}{Statistical calibration}

  Some parameter can be externally calibrated from long run data statistics.

  \bigskip

  Method of Simulated Moments (min. weighted $L^{2}$-norm):
  \begin{center}
    \includegraphics[width=5.2cm]{figures/calibration/model-fit/m1_gdp_model_fit}
    \
    \includegraphics[width=5.2cm]{figures/calibration/model-fit/markup_model_fit}
  \end{center}
  to pin down preference $(B, \sigma_{DM})$ and BJ contact rates $(\alpha_0, \alpha_1)$.

  \bigskip

      {\small\color{gray}
              \hfill Data: Lucas-Nicolini New M1 series; 
              \\ 
              \hfill Bank Prime Loan Rate/3 month TB rate
      }
\end{frame}

%% ==========================================================================
\begin{frame}{External validity}

  % \bigskip

  % \textbf{Model's prediction}:
  % \begin{center}
  %   \includegraphics[width=5.0cm]{figures/baseline/std_markups}
  %   \
  %   \includegraphics[width=5.0cm]{figures/baseline/mean_markup}
  % \end{center}

  % % % ~~~~~~~~~~~~~~~~~~~~~~~~~~~~~~~~~~~~~
  % % \break

  % % \textbf{Model's prediction}:


  % % \begin{itemize}
  % %   \item \alert{Observation 1}: \emph{Loan Rates Dispersion} increases with $\tau$
  % %   \item \alert{Observation 2}: \emph{Markup} (loan rate over deposit rate) increases with $\tau$
  % % \end{itemize}

  % % \bigskip

  % % Observation 1 and 2

  % \bigskip

  % \begin{center}
  %     $\Rightarrow$ \qquad \texttt{corr} ( Markup, loan rates dispersion )
  %     $ > 0$
  % \end{center}

  % % ~~~~~~~~~~~~~~~~~~~~~~~~~~~~~~~~~~~~~
  % \break

  % \begin{center}
  %   %\includegraphics[scale=0.45]{figures/data/MarkupGraph.png}
  %   \includegraphics[scale=0.22]{figures/data/national_correlation.png}
  %   \medskip
  % \end{center}

  

  % % ~~~~~~~~~~~~~~~~~~~~~~~~~~~~~~~~~~~~~
  \break

    $X \in \{ \textup{Markups SD}, \textup{ Markups CV} \}$

    \begin{table}
      \caption{$\text{corr}(X, \textup{Average Markup})$}
      \begin{center}
        \begin{tabular}{ l | c | r }
          \toprule
          \textbf{Data}             & $X=SD$ & $X=CV$
          %\cmidrule(r){1-2}
          \\
          \hline
          State, raw                & $0.40$ & $-0.15$ 
          \\ 
          \hline
          State, orthogonalized     & $0.41$ & $-0.05$ 
          \\ 
          \hline
          National, raw             & $0.75$ & $-0.86$ 
          \\
          \hline
          National, orthogonalized  & $0.51$ & $-0.58$
          \\
          \hline
          Model                     & $0.98$ & $-0.99$
          %\bottomrule
          \\
          \hline
          \hline
        \end{tabular}
      \end{center}
    \end{table}

    {
      \footnotesize
      \begin{itemize}
      \item Personal unsecured loan (Tier 1) rates (RateWatch, USA)
      %item $\approx 7 \times 10^{5}$ monthly (panel) observations over 20 years (2001-2020)
      \item National level statistics for dispersion vs average (percentage) markups
      \item Raw data and residualized (orthogonalized) markups
      \end{itemize}
    }

\end{frame}

%% ==========================================================================
%%      MILESTONE SLIDE - RESULTS
%% ==========================================================================
{
\setbeamercolor{background canvas}{bg=gray!40!black}
  \begin{frame}
    \begin{center}
      \bigskip
      \bigskip

      {\Huge\bfseries{\color{orange}Comparative Steady States}}
      \bigskip

    \end{center}
  \end{frame}
}

%% ==========================================================================
\begin{frame}{Comparative Steady States}
  \begin{itemize}
    \item Consider a set of economies, each distinguished by their long-run inflation rates, $\tau$

    \bigskip


    \item Questions to ask:
      \begin{itemize}
        \item Inflation tax and demand for loans

        \item Inflation tax and bank profits: intensive vs. extensive margins

        \item When are banks essential?

          \medskip

          \hfill c.f., \citet{Berentsen2007}

      \end{itemize}
  \end{itemize}
\end{frame}

%% ==========================================================================
% \begin{frame}{Comparative Steady States}
%   \framesubtitle{DM goods demand (credit line)}
%   \begin{center}
%     \includegraphics[scale=0.4]{figures/qstar_inflation}
%   \end{center}
% \end{frame}

%% ==========================================================================
\begin{frame}{Comparative Steady States}
  \framesubtitle{DM demand (credit line): partial equilibrium thinking}
  \begin{center}
    \includegraphics[scale=0.6]{figures/baseline/partial-equilibrium/loan_demand.png}
  \end{center}
  %\end{frame}

  %% ==========================================================================
  % \begin{frame}{Comparative Steady States}
  %   \framesubtitle{Discussion---Loan/Goods Demand Side}
  \begin{itemize}
    \item Higher inflation $\tau$, support of $F \sim i$ shifts right

    \item Also $z$ falls with higher $\tau$

    %\item \emph{Ceteris paribus}, DM-goods demand $q^{\star}(z,i; \tau)$ (conditional on meeting banks) falls since borrowing cost rises with $\tau$

    \item Loan demand $\xi^{\star}(z,\cdot; \tau)$ shifts right: Self-insurance by holding money more costly
  \end{itemize}
\end{frame}

%% ==========================================================================
\begin{frame}{Comparative Steady States}
  \framesubtitle{Lending rates}
  \begin{center}
    \includegraphics[scale=0.6]{figures/baseline/partial-equilibrium/pdf.png}
  \end{center}
  \begin{itemize}
    \item $F(i,\tau')$ FOSD $F(i,\tau)$, $\tau' > \tau$
    \item Mass shifts rightward: 
    
    \medskip
    
    Tension between markup (intensive margin) and potential loss of loan customers (extensive margin)
  \end{itemize}
\end{frame}

%% ==========================================================================
\begin{frame}{Comparative Steady States}
  \framesubtitle{Markups}
  \begin{center}
    \includegraphics[width=5.5cm, height=4.0cm]{figures/baseline/mean_markup.png}
    \
    \includegraphics[width=5.5cm, height=4.0cm]{figures/baseline/cv_markups.png}
  \end{center}
  Tension resolves:
  \begin{itemize}
    \item the equilibrium support of $F \sim i$ shifts right, and is wider

    \item the probability mass shift to the tails of $F$
  \end{itemize}
  %in order that firms are still meeting their equal-expected-profit condition.

    \bigskip

  \begin{itemize}
    \item At low enough $\tau$ banks tend to exploit the intensive (markup) channel

    \item At high enough $\tau$ extensive margin dominates
  \end{itemize}
\end{frame}

%% ==========================================================================
\begin{frame}{Comparative Steady States}
  \framesubtitle{DM banks' intensive vs. extensive profit}

  \begin{center}
    \includegraphics[width=5.5cm, height=4.0cm]{figures/baseline/partial-equilibrium/bank_expost_profit.png}
    \
    \includegraphics[width=5.5cm, height=4.2cm]{figures/baseline/banks_expected_profits.png}
  \end{center}
  % \end{frame}

  % %% ==========================================================================
  % \begin{frame}{Comparative Steady States}
  %   \framesubtitle{Discussion---Loan Supply Side}
  \begin{itemize}
    \item Higher inflation $\tau$, marginal cost $i_{d} \equiv (1+\tau)/\beta$ higher

    \item Eventually erodes expected profit as competition (via extensive margin) drives $\overline{i} \searrow i_{d}$.
    
    %\item Can show: markup 
      % \begin{itemize}
      %   \item \alert{Intensive margin}: Raise lending rate $i$ to keep profit margin, conditioning on loan demand in $i$ and ``being on the LHS of profit-max point''

      %   \medskip

      %    (But this drives customers away towards other (imperfectly) competing banks)

      %    \bigskip
      %    --- vs. ---
      %    \bigskip

      %   \item \alert{Extensive margin}: Lower lending rate $i$ to increase the frequency of contact with borrower.
      % \end{itemize}


    % \item Tension tends to resolve as follows:
    %   \begin{itemize}
    %     \item the equilibrium support of $F \sim i$ shifts right, and is wider

    %     \item the probability mass shift to the tails of $F$
    %   \end{itemize}
    %   in order that firms are still meeting their equal-expected-profit condition.
  \end{itemize}
\end{frame}



%% ==========================================================================
% \begin{frame}{Comparative Steady States}
%   \framesubtitle{Aggregate loans: Economy with BJ-Banks}
%   \begin{center}
%     \includegraphics[scale=0.25]{figures/aggregateloans_inflation}
%   \end{center}
%   Depends on policy $\tau$ via equilibrium:
%   \begin{itemize}
%     \item Real money demand: $z(\tau)$
%     \item Loan rate distribution: $F(i, z, \tau; \vec{\alpha})$
%     \item Loan demand schedule: $l^{\star}(i, z, \tau)$
%   \end{itemize}
% \end{frame}



%% ==========================================================================
\begin{frame}{Comparative Steady States}
  \framesubtitle{Welfare difference: Banks vs. no banks}
  \begin{center}
    \includegraphics[scale=0.5]{figures/baseline/difference_welfare.png}
        \begin{itemize}
          \item Low $\tau$: Banks exploit more \emph{intensive margin} markup channel
          %\item This destroys welfare-gain from intermediation role of banks.
          \item High $\tau$: Equilibrium loan price dispersion rises, loan quantity \emph{extensive margin} dominates, loan market more competitive, approaching BCW's original insight.
        \end{itemize}
  \end{center}
\end{frame}

\begin{frame}{Comparative Steady States}
    \framesubtitle{Welfare difference: Discussion}
  % ~~~~~~~~~~~~~~~~~~~~~~~~~~~~~~~~~~~~~~~~~~~~~~~~~~~~~~~~
  In BCW, banking improves welfare by insuring households against being stuck with unwanted cash in the DM (w.p. $1-n$).  

  \begin{itemize}
    \item The higher inflation the better it is (to a point)
    \item The lower aggregate demand ($n$), the better it is
  \end{itemize}
  %The extensions by Boel and Waller (2018,2019) add another benefit of banking.

  \bigskip

  Here, the banking system plays the same insurance role as in BCW.

  \bigskip

  But, borrowing is more expensive here due to \emph{imperfect competition}.

\end{frame}

%% ==========================================================================
%%      MILESTONE SLIDE - POLICY IMPLICATION
%% ==========================================================================
{
\setbeamercolor{background canvas}{bg=gray!40!black}
  \begin{frame}
    \begin{center}
      \bigskip
      \bigskip

      {\Huge\bfseries{\color{orange}
        Welfare and Policy Implications
        }
      }
      \bigskip

    \end{center}
  \end{frame}
}

\begin{frame}[allowframebreaks]{Policy Implications}

    Perfectly competititive banking ($\alpha_0 = \alpha_1 = 0$, $\alpha_2 = 1$):

    \begin{enumerate}
      \item Away from the Friedman rule, financial intermediation improves welfare.
      \item Due to payment of interest to depositors. Insuring idle funds.
    \end{enumerate}

    \bigskip

    Imperfectly competitive banks $\alpha_1 \in (0,1]$:
    \begin{enumerate}
      \item Away from the Friedman rule, financial intermediation does not necessarily improve welfare.
      \item Expected gain from insurance role lost through price dispersion ($\equiv$ banks extract borrowers' surplus)
    \end{enumerate}

    \bigskip

    Additional redistributive/liquidity policies can improve welfare towards \citet{Berentsen2007} ideal

  % %% ~~~~~~~~~~~~~~~~~~
  % \break

  % Discussion
  % \begin{itemize}
  %   \item Special case of perfect competition in banking sector yields the results of \citet{Berentsen2007}

  %   \item Second part: ``pedestrian conjecture'' earlier need not be self-evident truth!

  %   \item Moreover, in \citet{Berentsen2007}, timing of redistributive/liquidity policy is redundant. Only gain from banking is the insurance of idle funds through payment of deposit interest: net marginal benefit is just $(1-n)(1+i_d)$

  %   \item In our setting, depending on policies ($\gamma$ and transfer/timing $(\tau_b)$, and hence $z$, net marginal benefit from banking ($\Theta$) may be smaller than $(1-n)(1+i_d)$ - some lost to noisy search friction and equilibrium surplus extraction by banks with market power

  %   \item So additional redistributive/liquidity policies may help raise welfare towards the \citet{Berentsen2007} ideal
  % \end{itemize}
\end{frame}

% %% ==========================================================================
\begin{frame}[allowframebreaks]{Optimal Stabilization Policy}
  \framesubtitle{Active vs. Passive: $n$ aggregate demand shock}

  % \alert{Long run policy.} \ Interpret $\tau$ as desired inflation target $\equiv$ central bank has targeted price path
  % \begin{itemize}
  %   \item fixed (legislated, mandated)
  %   \item Cannot run Friedman rule, $\tau > 0$ (an institutional given)
  % \end{itemize}

  \citet{Berentsen-Waller2011}: In BCW, a temporary aggregate demand reduction ($n$) causes:
  \begin{itemize}
    \item deposits to rise and DM consumption/output to fall
    \item the deposit rate (equal to their loan rate) to fall as well
  \end{itemize}
  
  This is bad, because it diminishes the insurance the banking system can provide:
    \medskip
    \begin{itemize}
      \item[] Insurance (via bank deposits) has a low return precisely when households expect to need it.
    \end{itemize} 
    \bigskip

  Welfare can be improved by:
  \begin{itemize}
    \item Committing to an inflation target (really a price path)
    \item Injecting money in the DM when deposits are low (high $n$) and (if possible) removing it when deposits are high (low $n$)
  \end{itemize}

  % ~~~~~~~~~~~~~~~~~~~~~~~~~~~~~~~~~~~~~~~~~~~~~~~~~~~~~~~~
  \break
  Here, we deliberately turn off BW's mechanism by having $i_{d} = \tau/\beta -1$ constant and equivalent to BCW/BW's benchmark.

  \bigskip

  But, a similar policy can improve welfare. In our case, the optimal policy affects banks’ markups and takes into account markup endogeneity:

  \begin{itemize}
    \item \alert{Commitment to state-contingent transfers} in the DM $\tau_{1}(\omega)$.
    
    \item $\omega = \{\epsilon, n\}$ is an aggregate random variable. 
    
    \item Only transfers are permitted in the DM: $\tau_{1}(\omega) \geq 0$ for all $\omega$. (Anonymity; not incentive compatible.)
  
    \item State-contingent injections of liquidity in the DM followed by transfers/taxes in the CM (repo): 
      \[
        \tau_{2}(\omega) = -\tau_{1}(\omega),  \text{for all } \omega
      \]
    so that $\tau = M_{+1}/M = \phi/\phi_{+1}$ (price-level target is met in the long run).
    
    \item A stationary equilibrium with i.i.d. aggregate demand shocks, with five states $\{n_{1} < \cdots < n_{5}\}$.
  \end{itemize}

  % \bigskip
   % ~~~~~~~~~~~~~~~~~~~~~~~~~~~~~~~~~~~~~~~~~~~~~~~~~~~~~~~~
  %  \break

  % \alert{Short run policy.} \ Commitment to policy functions $\omega \mapsto (\tau_{1}, \tau_{2})(\omega)$
  % \begin{itemize}
  %   \item $\omega =\{\epsilon, n\}$ is a aggregate random variable
  %   \item demand-side stabilization
  %   \item any state-contingent injection of liquidity to DM agents will be undone in CM, i.e., $\tau_{2}(\omega)= -\tau_{1}(\omega)$; 
    
  %   i.e., a repo agreement where central bank sells money in DM and commits to buy back in CM
  %   \item w.l.o.g., we have $\tau_{1} = \tau_{b}$
  % \end{itemize}
  
  % \bigskip

  % \alert{Stationary equilibrium optimal policy.} \ Focus on equilibria where real balance $z$ is time invariant. 

  %  % ~~~~~~~~~~~~~~~~~~~~~~~~~~~~~~~~~~~~~~~~~~~~~~~~~~~~~~~~
  %  \break

   \alert{Active central bank} 
   \begin{equation*}
    \begin{split}
    \max_{\{ q_{b}^{0}(\omega),\ q_{b}^{1}(\omega),\tau_{b}(\omega)\}_{\omega \in \Omega}} 
    & 
    U\left(x\right)-x 
    - c(q_{s})
    \\
    &+ \int_{\omega\in\Omega}n\alpha_{0}\epsilon u\left[q_{b}^{0}\left(z;\rho,Z,\gamma,\tau_{b},\omega\right)\right]
    \psi\left(\omega\right)\text{d}\omega 
    \\
    &+ \int_{\omega\in\Omega}n
    \int_{\underline{i}}^{\overline{i}}
    \left[\alpha_{1}+2\alpha_{2}\left(1-F\left(i;z,\gamma,\tau_{b},\omega\right) \right)\right]
    \\
&\times \epsilon u\left[q_{b}^{1}\left(z;i,\rho,Z,\gamma,\tau_{b},\omega\right)\right]
    \text{d}F\left(i\right)\psi\left(\omega\right)\text{d}\omega
    \end{split}
    \label{eq:Active CB problem}
\end{equation*}
subject to:
\begin{itemize}
  \item optimal money demand (Euler condition) $\hookrightarrow z^{\star}(\tau)$
  \item credit search and bank profit max $\hookrightarrow F(\cdot, \tau, \omega)$
  \item goods markets clearing $\hookrightarrow x^{\star}, (q^{0}, q^{1})(z^{\star}(\tau), \omega)$
  \item Aggregate loan feasibility
  \item Policy commitment: 
    $\frac{\gamma-\beta}{\beta} = \tau + \tau_{1}(\omega) + \tau_{2}(\omega)$ and 
    $
    \tau_{1}(\omega) = -\tau_{2}(\omega)
    $
\end{itemize}

% ~~~~~~~~~~~~~~~~~~~~~~~~~~~~~~~~~~~~~~~~~~~~~~~~~~~~~~~~
\break

\alert{Passive central bank}

\begin{itemize}

  \item Policy constrained by $\tau_{1}\left(\omega\right) = \tau_{2}\left(\omega\right) = 0$ for all $\omega$. 

  \item The outcomes will be very similar to our deterministic, baseline SME.

\end{itemize}



  % ~~~~~~~~~~~~~~~~~~~~~~~~~~~~~~~~~~~~~~~~~~~~~~~~~~~~~~~~
  \break

  \begin{center}
    \includegraphics[scale=0.4]{figures/ramsey_n/tau_b.png}
  \end{center}

    % {\color{blue}
      Passive policy sets $\tau_{1}(n)$ for all $n$.
      % More transfers in high-demand state (relative to passive policy) ...
      \begin{itemize} 
        \item $\alpha_{2}=1$ here is the perfect-competition BCW case (we allow the deposit rate to vary)
      \end{itemize}
      \alert{Optimal active }: More transfers in high-demand state (relative to passive policy)
      \begin{itemize}
        \item when $\alpha_{2} =1$, policy to offset deposit rate movements
        \item when $\alpha_{2} <1$, policy to rectify/modify markup behaviors
      \end{itemize}
    % }


    % ~~~~~~~~~~~~~~~~~~~~~~~~~~~~~~~~~~~~~~~~~~~~~~~~~~~~~~~~
    \break
    \begin{center}
      \includegraphics[scale=0.3]{figures/ramsey_n/mean_i_states.png}
      \
      \includegraphics[scale=0.3]{figures/ramsey_n/mean_markups_states.png}
      \\
      \includegraphics[scale=0.3]{figures/ramsey_n/cv_markups_states.png}
      \
      \includegraphics[scale=0.3]{figures/ramsey_n/exante_cv_markups.png}
    \end{center}
    %Non-monoticity in active policy under imperfect competition:
    \begin{itemize}
      \item Higher $\tau_{1}(n)$ lowers real balances, inducing higher dispersion, but
      \item higher $\tau_{1}(n)$ directly lowers both dispersion and the average markup by reducing the monopoly rate.
    \end{itemize}

    % ~~~~~~~~~~~~~~~~~~~~~~~~~~~~~~~~~~~~~~~~~~~~~~~~~~~~~~~~
    \break

    \begin{center}
      \includegraphics[scale=0.35]{figures/ramsey_n/total_dm_goods_states.png}
      \ 
      \includegraphics[scale=0.35]{figures/ramsey_n/exante_utility.png}
    \end{center}

    This induces more (less) consumption and loans in state with more (less)
    active buyers, relative to passive policy regime.  

    \bigskip
    
    Thus active ``demand-side stabilization policy'' through liquidity provision results in higher ex-ante welfare for agents.

    % % ~~~~~~~~~~~~~~~~~~~~~~~~~~~~~~~~~~~~~~~~~~~~~~~~~~~~~~~~
    % \break

    % \begin{center}
    %   \includegraphics[scale=0.5]{figures/ramsey_n/exante_utility.png}
    % \end{center}

  
     


  % % ~~~~~~~~~~~~~~~~~~~~~~~~~~~~~~~~~~~~~~~~~~~~~~~~~~~~~~~~
  % \break

    % More results (alternative source of demand side shock):

    % \begin{itemize}
    %   \item We also consider taste shocks $\epsilon$ to $\epsilon u(q)$
      
    %   \item Similar punchlines, but more interesting shifts to supports of distributions $F$.
    % \end{itemize}
      
    % Here, two opposing forces:
    % \begin{itemize}
    %   \item When buyers have ex-post higher ($\epsilon$) valuation of DM goods, the marginal propensity to borrow increases. Banks tend to exploit this through creating noisier, more dispersed markups.
      
    %   \item But buyers also optimally demanding more real balances $z$. This tends to reduce the need to borrow, and hence banks incentives to mark up.
      
    %   \item With larger $\tau_{b}(\epsilon)$, $z$ tends to fall. This increases banks incentives to mark up. On the other hand, larger $\tau_{b}(\epsilon)$ also shifts markup distribution to the left and tightens its support.
    % \end{itemize}


\end{frame}

%% ============================================================================
\begin{frame}[allowframebreaks]{Punchline}

  % \begin{itemize}
  %   \item Empirical relevance:
  %     \begin{itemize}
  %       \item Model fits empirical money demand curve
  %       \item Implies positive correlation between loan rate markups and dispersion of loan rates distribution
  %     \end{itemize}
  %   \item Under equilibrium-determined banking market structure/competitiveness, financial intermediation not always ``essential'':
  %     \begin{itemize}
  %       \item At low inflation (away from Friedman rule), banks tend to exploit markup (intensive margin). This extracts surplus from households/borrowers and deposit on interest-earners not sufficient to overcome this welfare loss.
  %
  %       \item At higher inflation, there may be welfare gain from banking, provided that there are not too many active buyers requiring loans. So banks need to compete more for scarce borrowers to show up at their loans desks.
  %
  %       \item \emph{Conjecture}: There may be different timing of fiscal policies, in conjunction with monetary policy, that could induce the imperfectly competitive banking equilibrium outcome to be closer to the benchmark \citet{Berentsen2007} outcome.
  %     \end{itemize}
  % \end{itemize}

  \alert{Pass-through and welfare}

  \bigskip

  With information frictions, banks can be shown to be essential under perfect competition \citep{Berentsen2007}.

  % \bigskip

  % \emph{Perfect competition} in banking is one such strong assumption.

  \bigskip

  When market power of banks is endogenous to policy:

   \bigskip

    \begin{enumerate}
        \item equilibrium imperfect competition renders an otherwise \emph{essential} banking system detrimental, when 
        \begin{itemize}
                \item trend inflation is low, or
                \item aggregate demand is high
        \end{itemize}
        This will occur when the reduction in real balances (lower surplus share in monetary trades due to market power of lenders) outstrips the need for insurance.

        \item Pass-through of monetary policy (cost of funds variation) to lending interest rates

        \begin{itemize}
            \item positive relationship between the average markup and the dispersion of lending interest rates
        \end{itemize}
    \end{enumerate}

    \break

    %***********************************************************************

    \alert{Optimal stabilization policy}

    \bigskip

    Given a commitment to long-run trend inflation \dots

    \bigskip

    There is a role for stabilization policy, but it must internalize endogenous response of banking market power to policy:
    \bigskip

    Relative to passive policy, the optimal policy tolerates higher markups (and dispersion) when demand is low, and lowers them when it is high.

    



\end{frame}






%% ===========================================================================
%%      BIBTEX REFERENCING OUTPUT
%% ===========================================================================
\begin{frame}[allowframebreaks]{References}
  \begin{tiny}
    \printbibliography
    %\thebibliography{bjbanks}
  \end{tiny}
\end{frame}



%% ===========================================================================
%%         APPENDICES
%% ===========================================================================

\input{empirics-regression.tex}

\end{document}
